\documentclass[10pt,a4paper]{article}
\usepackage{ngerman}
\usepackage[utf8x]{inputenc}
\usepackage[T1]{fontenc}
\usepackage{lmodern}
\usepackage{marvosym}
\usepackage{ifpdf}
\usepackage[pdftex]{color}
\ifpdf
  \usepackage[pdftex]{graphicx}
\else
 \usepackage[dvips]{graphicx}
\fi

\pagestyle{empty}

\usepackage[scale=0.8]{geometry}
\setlength{\parindent}{0pt}
\addtolength{\parskip}{6pt}

\def\firstname{Patrick}
\def\familyname{Grüner}
\def\FileAuthor{\firstname~\familyname}
\def\FileTitle{\firstname~\familyname's Bewerbungsschreiben}
\def\FileSubject{Bewerbungsschreiben}
\def\FileKeyWords{\firstname~\familyname, Bewerbungsschreiben}

\renewcommand{\ttdefault}{pcr}
\hyphenation{ins-be-son-de-re}
\usepackage{url}
\urlstyle{tt}
\ifpdf
  \usepackage[pdftex,pdfborder=0,breaklinks,baseurl=http://,pdfpagemode=None,pdfstartview=XYZ,pdfstartpage=1]{hyperref}
  \hypersetup{
    pdfauthor   = \FileAuthor,%
    pdftitle    = \FileTitle,%
    pdfsubject  = \FileSubject,%
    pdfkeywords = \FileKeyWords,%
    pdfcreator  = \LaTeX,%
    pdfproducer = \LaTeX}
\else
  \usepackage[dvips]{hyperref}
\fi

%Liebherr yellow {254,180,1} black {0,0,0}
%Multivac {0, 105, 180}
%Grob {0,58,121}
%Rhode {0,60,124}
%BHS {0,80,148}
%Boeringer {1,51,103}
\definecolor{firstnamecolor}{RGB}{0,58,121}
\definecolor{familynamecolor}{RGB}{0,58,121}
\hypersetup{pdfborder=0 0 0}

\begin{document}
\sffamily   % for use with a résumé using sans serif fonts;
%\rmfamily  % for use with a résumé using serif fonts;
\hfill%
\begin{minipage}[t]{.6\textwidth}
	\raggedleft%
	{\bfseries {\color{firstnamecolor}\firstname}~{\color{familynamecolor}\familyname}}\\[.35ex]
	\small\itshape%
	Jahnstraße 5\\
	87700 Memmingen\\[.35ex]
	\Mobilefone~+49 176 47705895\\
	\Letter~\href{mailto:patrick.gmm91@gmail.com}{patrick.gmm91@gmail.com}
\end{minipage}\\[0.5em]
%
{\color{firstnamecolor}\rule{\textwidth}{.25ex}}
%
\begin{minipage}[t]{.5\textwidth}
	\raggedright%
	% {\bfseries {\color{firstnamecolor}
	\vspace*{1em}
	GROB-WERKE GmbH \& Co. KG\\[.35ex]
	% }}
	\small%
	Industriestraße 4  \\
	87719 Mindelheim
\end{minipage}
%
\hfill
%
\begin{minipage}[t]{.4\textwidth}
	\raggedleft % US style
	\today
	%April 6, 2006 % US informal style
	%05/04/2006 % UK formal style
\end{minipage}\\[1em]
\raggedright

{\bfseries \color{familynamecolor}Bewerbung als Applikationsingenieur für Bildverarbeitungssysteme (m/w/d)\\[1.5em]}
Sehr geehrte Frau Klaus,\\[1em]
%
auf der Suche nach einer neuen beruflichen Perspektive, habe ich Ihre spannende Vakanz des Applikationsingenieurs für Bildverarbeitungssysteme bei meiner Stellenrecherche entdeckt. Da ich als Softwareentwickler für Bildverarbeitungssysteme bereits viele Erfahrungen im Bereich Entwicklung, Konzeption und Umsetzung von Bildverarbeitungssystemen sammeln konnte, bin ich sehr davon überzeugt die optimale Besetzung für diese Stelle zu sein.

Seit mehr als fünf Jahren arbeite ich in der Entwicklung der Liebherr-Verzahntechnik GmbH im Bereich der Robotik-Automationssysteme. Dabei bin ich als projektleitender Softwareentwickler hauptverantwortlich für die Weiterentwicklung und Wartung der Bildverarbeitungssoftware LHRobotics.Vision, welche im Bereich der Vereinzelung von chaotischen Bauteilen mittels Industrieroboter eingesetzt wird. Hierbei arbeite ich erfolgreich in einem Team aus internen und externen Mitarbeitern an der Implementierung neuer Features, wie beispielsweise neuen Kalibrierverfahren, KI-Funktionen und Usability-Updates. Zusätzlich dazu übernehme ich die Konzeption und Parametrierung des Systems für Kundenprojekte und Machbarkeitsstudien und lege dabei auch selbst die eingesetzten 3D-Kameras beispielsweise der Firma Ensenso oder Zivid aus.

Meine Aufgaben bei Liebherr beschränken sich aber nicht nur auf den 3D-Bin Picking Bereich. Durch meine Expertise in der Bildverarbeitung  betreue ich mittlerweile den kompletten technischen Bereich der Bildverarbeitungssysteme in der Liebherr-Verzahntechnik GmbH, indem ich individuelle Systemarchitekturen aus Hard- und Software für Kundenprojekte auslege und umsetze. Die Bestandteile reichen hier von komplexen Vision Systemen, zum Beispiel Smartkameras der Firma Sensopart, bis hin zu Laserprofil- und Abstandssensoren der Hersteller Wenglor bzw. Sick. Hierbei stimme ich die Kundenanforderungen ab, berate frühzeitig zum Verhindern von Schwachstellen und kalkuliere die individuell entstehenden Kosten. Zusätzlich stimme ich mich mit den verschiedenen Fachbereichen aus der Mechanik, Elektrik und der SPS-Softwareentwicklung ab.

Wenn Sie einen Ingenieur für Bildverarbeitungssysteme suchen, der nicht nur einschlägige Erfahrung im Einsatz von fertigen 

Durch die eben beschriebenen Kompetenzen und Erfahrungen aus diversen Projekten, könnte ich in Ihrem Team von Anfang an ein wertvolles Mitglied sein. Gerne bringe ich auf diese Weise Ihr Unternehmen durch innovative technologische Lösungen nach vorne. Darüber hinaus bin ich überzeugt, dass Ihre Stelle als Vision Systems Engineer eine ideale Möglichkeit ist, meine beruflichen Interessen und Ziele weiterzuverfolgen. Deshalb würde ich mich sehr freuen, wenn wir zu einer Übereinkunft kommen und in Zukunft zusammenarbeiten.


Mit freundlichen Grüßen,\\[3em]

\includegraphics[scale=0.1]{C:/Users/Patrick/Documents/Bewerbung/Anlagen/Unterschrift}\\
{\bfseries \firstname~\familyname}\\
%
\vfill%
%{\slshape \bfseries Bewerbungsunterlagen}\\
% {\slshape Curriculum Vit\ae{}}
\end{document}
