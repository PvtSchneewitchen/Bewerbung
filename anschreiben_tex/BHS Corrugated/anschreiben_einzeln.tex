\documentclass[10pt,a4paper]{article}
\usepackage{ngerman}
\usepackage[utf8x]{inputenc}
\usepackage[T1]{fontenc}
\usepackage{lmodern}
\usepackage{marvosym}
\usepackage{ifpdf}
\usepackage[pdftex]{color}
\ifpdf
  \usepackage[pdftex]{graphicx}
\else
 \usepackage[dvips]{graphicx}
\fi

\pagestyle{empty}

\usepackage[scale=0.8]{geometry}
\setlength{\parindent}{0pt}
\addtolength{\parskip}{6pt}

\def\firstname{Patrick}
\def\familyname{Grüner}
\def\FileAuthor{\firstname~\familyname}
\def\FileTitle{\firstname~\familyname's Bewerbungsschreiben}
\def\FileSubject{Bewerbungsschreiben}
\def\FileKeyWords{\firstname~\familyname, Bewerbungsschreiben}

\renewcommand{\ttdefault}{pcr}
\hyphenation{ins-be-son-de-re}
\usepackage{url}
\urlstyle{tt}
\ifpdf
  \usepackage[pdftex,pdfborder=0,breaklinks,baseurl=http://,pdfpagemode=None,pdfstartview=XYZ,pdfstartpage=1]{hyperref}
  \hypersetup{
    pdfauthor   = \FileAuthor,%
    pdftitle    = \FileTitle,%
    pdfsubject  = \FileSubject,%
    pdfkeywords = \FileKeyWords,%
    pdfcreator  = \LaTeX,%
    pdfproducer = \LaTeX}
\else
  \usepackage[dvips]{hyperref}
\fi

%Liebherr yellow {254,180,1} black {0,0,0}
%Multivac {0, 105, 180}
%Grob {0,58,121}
%BHS {0,80,148}
\definecolor{firstnamecolor}{RGB}{0,80,148}
\definecolor{familynamecolor}{RGB}{0,80,148}
\hypersetup{pdfborder=0 0 0}

\begin{document}
\sffamily   % for use with a résumé using sans serif fonts;
%\rmfamily  % for use with a résumé using serif fonts;
\hfill%
\begin{minipage}[t]{.6\textwidth}
	\raggedleft%
	{\bfseries {\color{firstnamecolor}\firstname}~{\color{familynamecolor}\familyname}}\\[.35ex]
	\small\itshape%
	Jahnstraße 5\\
	87700 Memmingen\\[.35ex]
	\Mobilefone~+49 176 47705895\\
	\Letter~\href{mailto:patrick.gmm91@gmail.com}{patrick.gmm91@gmail.com}
\end{minipage}\\[0.5em]
%
{\color{firstnamecolor}\rule{\textwidth}{.25ex}}
%
\begin{minipage}[t]{.5\textwidth}
	\raggedright%
	% {\bfseries {\color{firstnamecolor}
	\vspace*{1em}
	BHS Corrugated Maschinen- und Anlagenbau GmbH\\[.35ex]
	% }}
	\small%
	Paul-Engel-Str. 1\\
	92729 Weiherhammer
\end{minipage}
%
\hfill
%
\begin{minipage}[t]{.4\textwidth}
	\raggedleft % US style
	\today
	%April 6, 2006 % US informal style
	%05/04/2006 % UK formal style
\end{minipage}\\[1em]
\raggedright

{\bfseries \color{familynamecolor}Bewerbung als Vision Systems Engineer (m/w/d))\\[1.5em]}

Sehr geehrte Frau Zeitler,\\[1em]
%
durch Herrn Fraaß von der Stratandnet GmbH bin ich auf Ihre interessante Stellenausschreibung des Vision System Engineers aufmerksam gemacht worden. Nach dreieinhalb spannenden und produktiven Jahren bei der Liebherr Verzahntechnik GmbH bin ich nun offen für neue berufliche Herausforderungen. Da die Stelle des Vision Systems Engineers vollkommen zu meinen beruflichen Tätigkeiten und Interessen passt und ihr Unternehmen auf mich einen sehr guten Eindruck macht, bin ich mir sicher, dass eine Zusammenarbeit zwischen Ihnen und mir außerordentlich erfolgreich sein würde.

Bei der Lieberr Verzahntechnik GmbH bin ich seit meiner Einstellung der zentrale Ansprechpartner für Bildverarbeitungsthemen sowohl im 2D- als auch im 3D-Bereich. Hierbei entwickle ich innerhalb von Kundenaufträgen verschiedenste Bildverarbeitungslösungen für Automationssysteme oder führe Machbarkeitsstudien bei Prototypen durch. Die Projekte belaufen sich dabei meist auf Themengebiete der Positions- und Lageerkennung von Objekten und deren Inspektion, wie beispielsweise Messungen oder Produktionsfehlererkennung. Innerhalb dieser Tätigkeiten konnte ich umfassende Erfahrungen mit diversen Bildverarbeitungssystem sammeln, unter anderem mit Smartkameras von Sensopart und Keyence oder mit komplexen 3D-Sensoren von Wenglor, Sick und IDS. Neben den Kundenprojekten leite ich auch interne Projekte im Bereich der Bildverarbeitung. Dieses Jahr schloss ich beispielsweise erfolgreich ein Projekt zur Standardisierung eines Bildverarbeitungssystems für Depalettieraufgaben ab.

Parallel zu diesen Ingenieurstätigkeiten in der Bildverarbeitung, bin ich zu gleichen Teilen als Softwareentwickler für die Weiterentwicklung und Wartung unserer hauseigenen Bildverarbeitungssoftware zuständig. Diese Software, genannt \glqq LHRobotics.Vision\grqq, ist in Kooperation mit dem Fraunhofer Institut entstanden und dient dem Zweck chaotisch angeordnete Bauteile aus einer Kiste zu befördern (\glqq Bin Picking\grqq). Dafür wird durch einen 3D-Sensor eine Punktewolke der Kiste erzeugt, in welcher die Software dann vorgegebene Bauteile erkennt und eine Robotertrajektorie zum Entnehmen dieser Teile berechnet. Seit meiner Einstellung verbessern wir diese Software als interdisziplinäres Team aus Liebherr- und Fraunhofer-Mitarbeiter hinsichtlich Performance und Usability und entwickeln Features wie beispielsweise neue Kalibriermethoden oder Tools zur Generierung von Sensordaten per Physiksimulation.

Durch die eben beschriebenen Kompetenzen und Erfahrungen aus diversen Projekten, könnte ich in Ihrem Team von Anfang an ein wertvolles Mitglied sein. Gerne bringe ich auf diese Weise Ihr Unternehmen durch innovative technologische Lösungen nach vorne. Darüber hinaus bin ich überzeugt, dass Ihre Stelle als Vision Systems Engineer eine ideale Möglichkeit ist, meine beruflichen Interessen und Ziele weiterzuverfolgen. Deshalb würde ich mich sehr freuen, wenn wir zu einer Übereinkunft kommen und in Zukunft zusammenarbeiten.

Mit freundlichen Grüßen,\\[3em]

\includegraphics[scale=0.1]{C:/Users/Patrick/Documents/Bewerbung/Anlagen/Unterschrift}\\
{\bfseries \firstname~\familyname}\\
%
\vfill%
%{\slshape \bfseries Bewerbungsunterlagen}\\
% {\slshape Curriculum Vit\ae{}}
\end{document}
